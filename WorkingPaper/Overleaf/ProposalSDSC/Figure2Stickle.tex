\documentclass{standalone}%scrreprt
\usepackage{pgfplots}
\usepackage{tikz}
\usepackage{tikz-network}
\usepackage{verbatim}
\usetikzlibrary{arrows,shapes,backgrounds,positioning}

\begin{comment}
Multilayer example following tikz-networks package using julia packages
\end{comment}
\usetikzlibrary{3d}

\begin{document}

\pagestyle{empty}
%\SetCoordinates[xAngle=90,yLength=1.2,xLength=.8]
\begin{tikzpicture}[multilayer=3d]

\begin{Layer}[layer=1]
\draw[thick,green!60] (-.5,-.5) rectangle (2.5,2);
\node at (-.5,-.5)[below right]{Genetic data};
\end{Layer}

\begin{Layer}[layer=2]
\draw[thick,blue!90] (-.5,-.5) rectangle (2.5,2);
\node at (-.5,-.5)[below right]{Phenotypic data};
\end{Layer}

\begin{Layer}[layer=3]
\draw[thick,orange!60] (-.5,-.5) rectangle (2.5,2);
\node at (-.5,-.5)[below right]{Isotope data};
\end{Layer}

\begin{Layer}[layer=4]
\draw[thick,red!80] (-.5,-.5) rectangle (2.5,2);
\node at (-.5,-.5)[below right]{Trophic data};
\end{Layer}

\begin{Layer}[layer=5]
\draw[thick,pink!100] (-.5,-.5) rectangle (2.5,2);
\node at (-.5,-.5)[below right]{Spatial data};
\end{Layer}

\Vertices{data/ml_vertices.figure2.csv}
\Edges{data/ml_edges.figure2.csv}
\end{tikzpicture}





%\begin{tikzpicture}[multilayer=3d]
%\begin{Layer}[layer=1]
%\draw[very thick] (-.5,-.5) rectangle (2.5,2);
%\node at (-.5,-.5)[below right]{Layer 1};
%\end{Layer}
%\Vertices[layer=1]{data/ml_vertices.csv}
%\Edges[layer={1,1}]{data/ml_edges.csv}
%\end{tikzpicture}
  


%\begin{tikzpicture}[multilayer=3d]
%\begin{Layer}[layer=1]
%\draw[very thick] (-.5,-.5) rectangle (2.5,2);
%\node at (-.5,-.5)[below right]{Layer 1};
%\end{Layer}
%\Vertices[layer=1]{data/ml_vertices.csv}
%\Edges[layer={1,1}]{data/ml_edges.csv}
%\end{tikzpicture}


%\begin{tikzpicture}[on grid,multilayer=3d]
%\draw[very thick] (-.5,-.5) rectangle (2.5,2);
%\Vertex[x=0.5,IdAsLabel,layer=1]{A}
%\Vertex[x=1.5,IdAsLabel,layer=1]{B}
%\Vertex[x=1.5,IdAsLabel,layer=2]{C}
%\Edge[bend=60](A)(B)
%\Edge[style=dashed](B)(C)
%\Edge(C)(C)
%\end{tikzpicture}


\end{document}